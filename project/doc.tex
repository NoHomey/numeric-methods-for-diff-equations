\documentclass[12pt]{article}
\usepackage[utf8]{inputenc}

\usepackage[T2A]{fontenc}
\usepackage[english,bulgarian]{babel}
\def\frak#1{\cal #1}

\usepackage{amssymb,amsmath}
\usepackage{graphicx}
\usepackage{shuffle}
\usepackage{alltt}
\usepackage{enumerate}
\newtheorem{theorem}{Теорема}%[section]
\newtheorem{problem}{Задача}%[section]
\newtheorem{remark}{{Забележка}}%[section]
\newtheorem{example}{Пример}%[section]
\newtheorem{lemma}{{Лема}}%[section]
\newtheorem{proposition}{Твърдение}%[section]
\def\proof{\textbf {Доказателство: }}%[section]
\newtheorem{corollary}{Следствие}%[section]
\newtheorem{fact}{Факт}%[
\newtheorem{definition}{Дефиниция}%[section]

\author{Иво Стратев}
\title{Проект по ЧМДУ-практикум 2019/2020г. Вариант 3}

\begin{document}
\maketitle

\begin{problem}
Разглеждаме следната диференциална задача:

\begin{align}
u''(x) - x^2u(x) = x^2\ln(x) + x^{-2} \;, \; x \in (1, 2) \\
u(1) = 0 \\
u'(2) = \frac{1}{2}
\end{align}

Съставяме диференчна схема, за която искаме локалната грешка на апроксимация да е \(O(h^2)\),
където \(h\) е стъпката.

\vspace{0.5cm}

Въвеждаме мрежа от \(n + 1\) точки в интервала \([1, 2]\).
Нека \(h = \frac{1}{n}\).
Нека \(x_i = 1 + (i - 1)h\) за \(i \in \{1, 2, \dots, n + 1\}\). 
Нека \(y_i \approx u(x_i)\). 
\end{problem}

\vspace{0.5cm}

В уравнение \((1)\) апроксимираме \(u''(x_i)\) с
\begin{align*}
\displaystyle\frac{y_{i + 1} - 2y_i + y_{i - 1}}{h^2}
\end{align*}

с ЛГА \(O(h^2)\).

И така последователно получаваме

\begin{align*}
\displaystyle\frac{y_{i + 1} - 2y_i + y_{i - 1}}{h^2} - x_i^2y_i = x_i^2\ln(x_i) + x_i^{-2}
\end{align*}

\begin{align*}
y_{i - 1} -(2 + h^2x_i^2)y_i + y_{i + 1} = h^2x_i^2\ln(x_i) + h^2x_i^{-2}
\end{align*}

за \(i \in \{2, 3, \dots, n\}\).

\vspace{0.5cm}

Граничното условие \((2)\) се апроксимира точно без грешка от \(y_1 = 0\).

\vspace{0.5cm}

За апроксимация на граничното условие \((3)\) с ЛГА \(O(h^2)\)
въвеждаме временна извън конурна точка \(x_{n + 2} = x_{n + 1} + h = 2 + h\).
Допускаме, че уравнението \((1)\) се удовлетворява в \(x_{n + 1}\).
Апроксимираме \(u'(2)\) с централна разлика с ЛГА \(O(h^2)\).

Последователно получаваме:

\begin{align*}
y_{n} -(2 + h^2x_{n + 1}^2)y_{n + 1} + y_{n + 2} = h^2x_{n + 1}^2\ln(x_{n + 1}) + h^2x_{n + 1}^{-2} \\
\displaystyle\frac{y_{n + 2} - y_{n}}{2h} = \frac{1}{2}
\end{align*}

\begin{align*}
y_{n} -(2 + h^2 2^2)y_{n + 1} + y_{n + 2} = h^2 2^2\ln(2) + h^2 2^{-2} \\
y_{n + 2} = y_{n} + h
\end{align*}

\begin{align*}
2y_n -2(1 + 2h^2)y_{n + 1} + h = 4h^2\ln(2) + \frac{h^2}{4}
\end{align*}

\begin{align*}
y_n -(1 + 2h^2)y_{n + 1} = h^2\left(\ln4 + \frac{1}{8}\right) -\frac{h}{2}
\end{align*}

Тоест крайната диференчна схема е
\begin{align*}
y_1 = 0 \\
y_{i - 1} -(2 + h^2x_i^2)y_i + y_{i + 1} = h^2\left(x_i^2\ln(x_i) + x_i^{-2}\right) \; , \; i \in \{2, 3, \dots, n\} \\
y_n -(1 + 2h^2)y_{n + 1} = h^2\left(\ln4 + \frac{1}{8}\right) -\frac{h}{2}
\end{align*}  

\end{document}